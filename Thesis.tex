\documentclass{mimosis}

\usepackage{metalogo}

%%%%%%%%%%%%%%%%%%%%%%%%%%%%%%%%%%%%%%%%%%%%%%%%%%%%%%%%%%%%%%%%%%%%%%%%
% Some of my favourite personal adjustments
%%%%%%%%%%%%%%%%%%%%%%%%%%%%%%%%%%%%%%%%%%%%%%%%%%%%%%%%%%%%%%%%%%%%%%%%
%
% These are the adjustments that I consider necessary for typesetting
% a nice thesis. However, they are *not* included in the template, as
% I do not want to force you to use them.

% This ensures that I am able to typeset bold font in table while still aligning the numbers
% correctly.
\usepackage{etoolbox}

\usepackage[binary-units=true]{siunitx}
\DeclareSIUnit\px{px}

\sisetup{%
  detect-all           = true,
  detect-family        = true,
  detect-mode          = true,
  detect-shape         = true,
  detect-weight        = true,
  detect-inline-weight = math,
}

%%%%%%%%%%%%%%%%%%%%%%%%%%%%%%%%%%%%%%%%%%%%%%%%%%%%%%%%%%%%%%%%%%%%%%%%
% Hyperlinks & bookmarks
%%%%%%%%%%%%%%%%%%%%%%%%%%%%%%%%%%%%%%%%%%%%%%%%%%%%%%%%%%%%%%%%%%%%%%%%

\usepackage[%
  colorlinks = true,
  citecolor  = RoyalBlue,
  linkcolor  = RoyalBlue,
  urlcolor   = RoyalBlue,
  unicode,
  ]{hyperref}

\usepackage{bookmark}

%%%%%%%%%%%%%%%%%%%%%%%%%%%%%%%%%%%%%%%%%%%%%%%%%%%%%%%%%%%%%%%%%%%%%%%%
% Bibliography
%%%%%%%%%%%%%%%%%%%%%%%%%%%%%%%%%%%%%%%%%%%%%%%%%%%%%%%%%%%%%%%%%%%%%%%%
%
% I like the bibliography to be extremely plain, showing only a numeric
% identifier and citing everything in simple brackets. The first names,
% if present, will be initialized. DOIs and URLs will be preserved.

\usepackage[%
  autocite     = plain,
  backend      = bibtex,
  doi          = true,
  url          = true,
  giveninits   = true,
  hyperref     = true,
  maxbibnames  = 99,
  maxcitenames = 99,
  sortcites    = true,
  style        = numeric,
  ]{biblatex}

%%%%%%%%%%%%%%%%%%%%%%%%%%%%%%%%%%%%%%%%%%%%%%%%%%%%%%%%%%%%%%%%%%%%%%%%
% Tables
%%%%%%%%%%%%%%%%%%%%%%%%%%%%%%%%%%%%%%%%%%%%%%%%%%%%%%%%%%%%%%%%%%%%%%%%
%
% DEFINED BY RUI - NOT IN THE TEMPLATE

\usepackage{tabulary}
\usepackage[export]{adjustbox} % To put borders around some figures

\usepackage[inline]{enumitem}

%\input{bibliography-mimosis}
\addbibresource{Thesis.bib}

%%%%%%%%%%%%%%%%%%%%%%%%%%%%%%%%%%%%%%%%%%%%%%%%%%%%%%%%%%%%%%%%%%%%%%%%
% Fonts
%%%%%%%%%%%%%%%%%%%%%%%%%%%%%%%%%%%%%%%%%%%%%%%%%%%%%%%%%%%%%%%%%%%%%%%%

\ifxetexorluatex
  \setmainfont{Minion Pro}
\else
  \usepackage[lf]{ebgaramond}
  \usepackage[oldstyle,scale=0.7]{sourcecodepro}
  \singlespacing
\fi

\renewcommand{\th}{\textsuperscript{\textup{th}}\xspace}

%\newacronym[description={Principal component analysis}]{PCA}{PCA}{principal component analysis}
\newacronym                                            {sdn}{SDN}{Software-defined networking}
\newacronym                                            {api}{API}{Application Programming Interface}
\newacronym                                            {vpcs}{VPCS}{Simple Virtual PC Simulator}
\newacronym                                            {dhcp}{DHCP}{Dynamic Host Configuration Protocol}
\newacronym                                            {rest}{REST}{Representational state transfer} % TODO put it the glossary
\newacronym                                            {nic}{NIC}{network interface controller} % TODO put it the glossary
\newacronym                                            {mtu}{MTU}{maximum transmission unit} % TODO put it the glossary
\newacronym                                            {nat}{NAT}{network address translation}

\newglossaryentry{LaTeX}{%
  name        = {\LaTeX},
  description = {A document preparation system},
  sort        = {LaTeX},
}

\newglossaryentry{Real numbers}{%
  name        = {$\real$},
  description = {The set of real numbers},
  sort        = {Real numbers},
}

% TODO Add entry for REST
% TODO Add entry for WebSockets

\makeindex
\makeglossaries

%%%%%%%%%%%%%%%%%%%%%%%%%%%%%%%%%%%%%%%%%%%%%%%%%%%%%%%%%%%%%%%%%%%%%%%%
% Incipit
%%%%%%%%%%%%%%%%%%%%%%%%%%%%%%%%%%%%%%%%%%%%%%%%%%%%%%%%%%%%%%%%%%%%%%%%

%%%%%%%%%%%%%%%%%%%%%%%%%%%%%%%%%%%%%%%%%%%%%%%%%%%%%%%%%%%%%%%%%%%%%%%%
% Rui's LaTeX and/or editor hacks and notes go here.
% e.g. useful regex to search for in the file.
% Things we want to look for:
% Comments starting with the "TODO" string
% Lines (in the text file) that contain more than two sentence.
% Take into account that some sentences end in doube-quotes
% (with the perid . before it)
%
% TODO Use the oxford comma?
% TODO see how dates are written in English across the whole document
% TODO check if there are i.e. with a "comma" after it
% TODO check how many text between parenthesis and m-dashes is there
% TODO check for double-quote american errors, i.e. commas out of "",
% TODO check for <CAPS LOCK STUFF INSIDE ST/GT SIGNS>
% TODO see how NOVAthesis does the side by side figure thingy
%%%%%%%%%%%%%%%%%%%%%%%%%%%%%%%%%%%%%%%%%%%%%%%%%%%%%%%%%%%%%%%%%%%%%%%%

\title{Usage of network emulators for the benefit of teaching and learning}
%\subtitle{A minimal, modern \LaTeX{} package for typesetting your thesis}
\author{Rui Miguel Carrilho Carvalho}

\begin{document}

\frontmatter
  \include{Sources/Title}
  % !TEX root = ../Thesis.tex
% !TEX spellcheck = en-US

\cleardoublepage\thispagestyle{plain}

\textbf{\Large Abstract}

In recent years computer networks emulators have been developed, mostly relying on virtualization technologies, which allow performing experiments that reproduce in a quite realistic way what happens in a real network.

In parallel, not only do personal computers (PCs) have nowadays much more resources (memory capacity, processor speed, etc.) than a few years ago, but Internet connectivity is increasingly ubiquitous, facilitating contact with powerful infrastructures, in the so-called \emph{cloud} or in institutional data-centers, which can be relied upon to set up experiments with emulators, in cases where these require more resources than those available on a PC.

This thesis critically and comparatively assesses how two paradigmatic examples of modern emulators, GNS3 and Kathará, provide answers to existing problems in teaching, learning, and experimental research in computer networks, namely whether it is possible, thanks to them, to mitigate or eliminate the dependency on a laboratory with physical equipment to perform certain practical exercises.

\textbf{Keywords:} Computer networks; emulators; virtualization; education.


  \tableofcontents

\mainmatter

  % !TEX root = ../Thesis.tex
% !TEX spellcheck = en-US

\chapter{Introduction}
\label{ch:introduction}

This first chapter introduces the challenge of using virtual, fully software-based alternatives to computer network labs in education, with special emphasis on the undergraduate and graduate levels of the university. % TODO should university by capitalized here?
It gives reasons to do so, and provides an overview of the work that has already been done both in development of those kinds of tools and in their current introduction, and study thereof, in existing universities and courses.
Furthermore, it explains the motivation behind the main work performed in the present document, which consists in studying and comparing different existing software solutions for simulating and emulating computer networks and, both theoretically and practically, justifying how each one can or cannot be a got fit for particular use-cases of different courses or subjects inside computer networks studying.

% It also proposes the definitions that, though not universal or unique, help precise differences between those software solutions, namely two main concepts: simulation and emulation. % TODO universal NOR unique?
% Together with the last point, this chapter also explains the reason why this thesis is centered, starting from its title, in ``emulators'' more than on ``simulators''.
% Finally, it gives a very brief overview of the work that has already been done in the field of computer networks simulators and emulators, and, orthogonally, in the replacement of laboratories with (physical) hosts and routers/switches in schools and universities by simulated or emulated (virtual) solutions.

\section{Motivation and problem statement}
\label{sec:motivation}

% Teaching of computer networks courses and subjects, to fulfill the desired properties of a ``hands-on approach'' in engineering, one which enables putting in practice the knowledge acquired in the theoretical component of subjects, typically require performing two kinds of exercises: i. on the application layer, using APIs that give access to the OS's network-stack and the host interfaces; and, ii. on ``lower layers'' of the stack, oftentimes using physical labs, rooms equipped with networking gear, like switches and routers. % TODO glossário: APIs; citar referências às ditas "propriedades desejáveis" % TODO apagar
Teaching of computer networks courses and subjects, to fulfill the desired properties of a ``hands-on approach'' in engineering---one which enables putting in practice the knowledge acquired in the theoretical component of subjects--- typically requires performing two kinds of exercises: % TODO glossário: APIs; citar referências às ditas "propriedades desejáveis"
  \begin{enumerate*}[label=(\roman*), itemjoin={{, }}, itemjoin*={{, and }}]
  \item on the application layer, using APIs that give access to the OS's network-stack and the host interfaces
  \item on lower layers of the stack, oftentimes using physical labs, rooms equipped with networking gear, like switches and routers.
  \end{enumerate*}
The existence of such laboratories is, even today, standard practice in many institutions of higher education.
FCT/NOVA, which is not an exception, has a room with routers and switches for performing those practical exercises.
% This thisis' goal is to survey, study, and compare software based solutions to address the latter as well as possible.
% The existence of those laboratories is, even today, standard practice in many institutions of higher education. FCT/NOVA, which is not an exception, has a room with routers and switches for performing those practical exercises. % TODO inserir aqui fotografia do laboratório da FCT/NOVA?

Despite its virtues, this method for interacting with switching and routing, its protocols, the impacts of different parameters and node behavior on the behavior of a complex application that relies on the network, raises problems such as the cost and fast obsolescence~\cite{automaticnetconfiggns} of equipments, the demand for individual presence on the laboratory (at least for manipulating physical links and network interfaces) for exercise elaboration, the possible damage due to misuse~\cite{teachinginovation} or simple wear-and-tear, the time and effort to prepare and setup for given exercises, which may ``destroy'' the setup for other exercises, etc. % TODO avoid repeating "exercise" so much here

Aiming to address these problems, not exclusive from further or higher education, common to vocational education on high schools and ``\emph{politécnicos},'' and ``courses'' of the industry itself, of which Cisco's CCNA~\cite{ccna} is a prominent example, or even distance learning~\cite{networkvirtwithgns}, tools for emulating and/or simulating networks and networking equipment have been developed for years. % TODO replace "extending". o "vocational education" veio do linguee
% In particular, emulators, much more powerful, flexible, and easier to use in the past due to the recent improvement in computer resources available, and also software technologies, as will be seen later, can provide real interactions...
These tools can, through the usage of several techniques, surpass some (if not all) of the aforementioned disadvantages.
Critically evaluate whether one or more than one, used in a complementary fashion, of the existing simulators/emulators can live up to that ``promise'' is, in a few words, the goal of this work.

% \subsection{The lab on a laptop}
% \label{subsec:thelabonalaptop}

% A very important question that must be taken into account is that the functional hyper-realism of emulators comes at a cost of a lot of resources.
% One or more machines must run a potentially high number of processes where the code that typically runs inside \textbf{each slice} of the stack for \textbf{each node} of the topologies is executed, in a highly concurrent fashion.
% The performance of the hardware can degrade and the observed behavior of the topology may look very unrealistic compared to ``the real thing,'' as obviously there may be less aggregate computing power

\section{Structure of the thesis}
\label{sec:structure}

% end of chapter
 % Chapter 1
  \include{Sources/RelatedWork/ChRelatedWork} % Chapter 2
  % !TEX root = ../Thesis.tex
% !TEX spellcheck = en-US

\chapter{GNS3}
\label{ch:gns3}

GNS3, whose original name was ``Graphical Network Simulator-3,'' is a software project, comprising several distinct components and, despite the ``simulator'' in its name, \emph{as a whole}, it falls under the category of emulators (cf.~\ref{sec:emulationsimulation}): its operation runs real code accross all the layers of the stack. % TODO try to (either with a source, or speculating) relate the name with ns-3
% TODO also: cite the source for the name
% By ``as a whole,'' it is meant that, as shall be seen later, although some of its components, like the Dynamips program, are emulators in a very strict sense---i.e. its purpose is to run real machine code on a different hardware architecture (than its native one)---, it differentiates itself, on a high-level perspective, from a simulator which is a program designed to execute a mathematical model, processing modeled events as internal data-structures with a collection of preset algorithms that somehow mimic (a part of) the reality. % TODO: delete?

Note that a necessary step towards the goal of this this thesis is to systematize a technical description of the architecture and functional underpinnings of the GNS3 system.
How does it interact with the hardware and software it is running in, how much resources does it take to work with GNS3 according to the ``mode'' it is running in, etc., as that is an aspect that is yet to be further explored for, at least, the following reasons:  % TODO improve the writing how is etc written in Engilsh?
on the one hand, the consulted academic material (i.e. research papers, mostly) is very brief and omissive in regards to the design, architecture and implementation of GNS3, and so is the official website and documentation; on the other hand, many interesting details are in the ``paraofficial'' videos available via YouTube, mostly by David Bombal\footnote{\url{https://www.youtube.com/channel/UCP7WmQ_U4GB3K51Od9QvM0w}}, namely a comprehensive overview of the architecture (compared with the functionality) of GNS3 by its creator, Jeremy Grossmann, and essentially are not anywhere else, at least from authoritative sources. % TODO same as the previous "sentence". Cite the videos (how and which ones?)

% TODO add a figure (a screen shot) showing the "official" submitter of GNS3 videos (and maybe the channel and/or links from the gns3 official website)

% end of intro

% Section "GNS3's purpose and raison d'être"
\section{GNS3's purpose and \emph{raison d'être}}
\label{sec:gns3why}

The GNS3 project was co-created by Jeremy Grossmann at the University, the Ecole informatique Epitech, as part of the \emph{EPITECH Innovative Project (EIP)}\footnote{\url{https://eip.epitech.eu/2013/gns3/en/index.html}, accessed on December 2019}.
There is also a blog\footnote{\url{http://gns3.blogspot.com/2007/}, accessed on December 2019}, whose first posts date back to 2007, that announces the first ``beta'' releases of the software and gives the details about the inception of GNS3, and also lists the original developers of the project.

It may be speculated that GNS3's first ``appeal'' was supplying students of Cisco certifications with a self-study tool more powerful that anything before, one that allows for the creation of arbitrary network topologies and practice their skills with the same software stack that is used in real Cisco devices and the hosts connected to them---from the operating system, up until application--level network utilities---, without having to use the expensive official solutions for that purpose, or depending on a real, physical networking laboratory.
However, despite having kept a strong relation with training for Cisco CCNA and related certifications, the magnitude of the project, part of which comes from an inherent extensibility, has made it suitable for a myriad of use-cases.

In the industry, GNS3 naturally can come handy as a tool to prototype and test a topology for a real organization network, in the context of the practice of a network professional.
But also, in the age of the DevOps philosophy---a set of practices and methodologies/philosophies for faster delivery of applications and services---, of which \emph{infrastructure as code} is one of the culprits~\cite{awswhatisdevops}, it can be of value for software engineers to test distributed applications that depend on certain network behaviors. % TODO try to find a good list of use cases

% At the moment of the writing of this thesis, version 2.2 has recently came out\footnote{\url{https://github.com/GNS3/gns3-gui/releases/tag/v2.2.0}}.
% This version brings a lot of improvements and new features, like the new web UI or ``physical'' link state detection for QEMU nodes~\cite{releasenotesgns3v22}, and at the same time some of the ways GNS3 has been typically used are changing---such is the case of the discouragement of the usage of Dynamips and VPCS (in detriment of virtualized versions of IOS or others, or Docker containers for ``emulating'' hosts, respectively)~\cite{ytdynamipsvpcs}.


% Section "Building blocks. The programs inside GNS3"
\section{Building blocks. The programs ``inside'' GNS3}
\label{sec:gns3buildingblocks}

In an very simplified way, GNS3 is a conjunction of UI/client tools (the official GUI, the new web UI, or any custom program that ``speaks'' the documented public \acrshort{REST} API), a powerful distributed orchestrator (the \emph{controller} in the server), and a set of integrations (the so-called ``compute nodes'') with virtualization and hardware emulation tools from ``the outside,'' i.e. that do not belong to the GNS3 project, like KVM, Docker, or QEMU, to provide a way to describe a topology of interconnected computing and routing nodes---that is, a computer network---, including firewalls and NAT devices, control their behavior and launch administrative tools (like \texttt{telnet} sessions). % TODO cite documentation of the REST API. Make sure we elaborate on "computes". How is so-called written - is an hyphen used?

% What follows is a description of what are those elements and what they do work internally.
% How they integrate with GNS3 or vice-versa, and also how they interact with each other, is described in~\ref{sec:gns3architecture}.

\begin{figure}
  \centering
  \includegraphics[width=0.8\textwidth]{download-gns3}
  \caption{The download screen on the official GNS3 website}
  \label{fig:download-gns3}
\end{figure}

When GNS3 is installed downloading one of its desktop distributions---available for Windows, macOS, and GNU/Linux, like the screen shot~\ref{fig:download-gns3} shows---, a user is installing multiple ``programs'' (or applications, whichever is the preferred denomination), implemented in disjunct codebases, and, given that they are \emph{free and open source} projects~\cite{gplv3}, hosted in GitHub, are easily accessible---and developers can enhance features and provide bugfixes.
Those are the essential building blocks of GNS3 and table~\ref{tab:gns3components} can serve a summary of what those pieces are.
It's worth noting that they are all under the GNS3 organization in GitHub\footnote{\url{https://github.com/gns3}}.

\begin{table}
  \centering
  \begin{tabulary}{0.9\textwidth}{lLL}
    \toprule
      \textbf{Part}  & \textbf{Role}                                                       & \textbf{Source code repository}\\
    \midrule
      GNS3 GUI       & A desktop application that runs on a graphical OS                   & \url{https://github.com/GNS3/gns3-gui}\\
      Dynamips       & A MIPS emulator, used by the \emph{backend} to run (old) IOS images & \url{https://github.com/GNS3/dynamips}\\
    \bottomrule
  \end{tabulary}
  \caption{%
    A list of the intrinsic parts of GNS3, constituting separate codebases
  }
  \label{tab:gns3components}
\end{table}


\subsection{GNS3 GUI}
\label{subsec:gns3gui}

Interaction between the end-user and GNS3 is usually---though, as will be clear, not necessarily---done in a graphical environment.
A GNS3 project, called a \emph{topology}, is constantly opened on one single window (per running instance of the application) and is graphically represented in the main section of the window.

Even though the GUI is not the authoritative source of truth for a topology (cf.~\ref{sec:gns3architecture}), it can be used as interface to all of GNS3's functionality: edit the topology itself (adding or removing nodes, changing links), performing actions on the nodes (like starting or stopping a host or router), or using helpers to launch consoles automatically connected via \texttt{telnet} or SSH.
In the section dedicated to the usage of GNS3~\ref{sec:gns3inaction}, it will be shown how this is done from a user's perspective.
Many GUI clients, on different hosts (e.g. laptops), can be editing the same topology at the same time. % TODO try to capture screenshots of this happening (use VMs)

\begin{figure}
  \centering
  \includegraphics[width=0.8\textwidth]{gns3-empty-topology}
  \caption{An empty GNS3 topology shown in the GUI}
  \label{fig:gns3-empty-topology}
\end{figure}

\subsection{Dynamips}
\label{subsec:gns3dynamips}

The Dynamips emulator is a standalone program, written in C, that, usually, comes distributed together with the whole GNS3 package.
It is an emulator for a MIPS processor and was the original--single way to run the software of the Cisco nodes of the topologies created with GNS3.


\subsection{GNS3 server}
\label{subsec:gns3server}


\subsection{GNS3 VM}
\label{subsec:gns3vm}

% end of section gns3buildingblocks


% Section "General architecture"
\input{Sources/Gns3/GeneralArchitecture}

\section{GNS3 in action}
\label{sec:gns3inaction}

% end of section gns3inaction

\section{Performance and resources considerations}
\label{sec:gns3performance}

% end of section gns3performance

% end of chapter
 % Chapter 3
  % !TEX root = ../Thesis.tex
% !TEX spellcheck = en-US

\chapter{Kathará}
\label{ch:kathara}

Kathará~\cite{kathara} is a new implementation, created at the Roma Tre University using more modern technologies, of Netkit, a network emulator developed at the institution years before.
It was made with the purpose of introducing functionalities of emulating SDN and NFV based networks, as well as leveraging the P4 language for programming forwarding-planes.
Still, Kathará---as explicitly stated in the previously cited presentation article---is fully compatible with Netkit, having a similar interface and conceptual philosophy than its predecessor, in particular fully supporting topologies based in ``standard'' routing devices and generic hosts.

\section{What is it}
\label{sec:katharawhatis}

\section{Architecture}
\label{sec:katharaarchitecture}

\section{GUI to setup projects}
\label{sec:katharagui}

\section{Setup of the topology and results}
\label{sec:katharatopologyexample}

% end of chapter
 % Chapter 4
  % !TEX root = ../Thesis.tex
% !TEX spellcheck = en-US

\chapter{Comparative analysis}
\label{ch:comparative}

% Section "Functionality"
\section{Functionality}
\label{sec:comparativefunctionality}

Functionality, in particular in the emulated network-nodes,\footnote{By contrast with end-nodes, which no matter the technology are expected to be, in some sense, generic virtual, isolated hosts running a typical end-host operating system, and therefore are of least concern in this sense} is defined, in this document, in two senses:
\begin{enumerate}
  \item From a \emph{higher-level} perspective, the possibility to run \emph{generically} existing algorithms and protocols, exchanging packets in the respective layers according to the standard way to do so.
  \item From a \emph{lower-level} point of view, the ability to run vendor-specific software, with all of its particular attributes, maybe proprietary optimizations, augmented headers, or any kind of functionality that a particular ``brand'' of networking software may offer.
\end{enumerate}

In terms of the \emph{higher-level functionality}, most general-purpose emulators, which GNS3 and Kathará are at heart, don't differ much.
Both have a way to define arbitrary topologies and store them in the filesystem and the topologies describe nodes, default configuration of the nodes, and hosts represent ``computers'' that, according to the specifics of the software/firmware they are running and number of interfaces serve as switches, routers, end-hosts running application software (client/server, P2P, etc.), or even other kinds of nodes seen in real-world networks, often called middleboxes, not studied in the present work.

However, if, for example, running Cisco software is a requirement (as is the case for the company's official certifications), Kathará itself cannot accommodate it.
On the contrary, there aren't officially supported GNS3 appliances, running on any kind of platform, to run Quagga out of the box.
This is to show that, \textbf{in an academic context}, where protocols and algorithms are the subject of the study, and not vendor-specific details, \textbf{Kathará has the advantage of doing without pricey licenses and vendor lock-in}.

The degree of flexibility of the two emulators is different, due to reasons probably obvious by now.
On the one hand, GNS3 is ``a suite of \emph{emulation methods}'' (and even more than that, giving the diversity of software packages that are inside a normal installation of the software on a desktop), and offers the ability to emulate nodes using large span of technologies that don't typically work with each other by making ensuring that one single software package, uBridge, is pluggable to each of them, being responsible to handling the transmission of the traffic, in a way that each node's emulation method is never aware of the emulation method on the other end of a virtual link.
On the other hand, Kathará doesn't use any application-level data-link emulation, and requires that every kind of software, be it for switching/routing nodes, middleboxes, is able to communicate using Docker's virtual interfaces.
Both Kathará and GNS3 provide a way to connect an interface on the host operating system to a virtual interface on its nodes.
This, if thought in conjunction with the way hypervisors like VMware Workstation allow to create virtual switches and interfaces in the host computer, allows for virtually unlimited combination of topologies across both emulators, using physical interfaces, and even mixing them with other ones on the Internet.

% end of section comparativefunctionality


% Section "Non-functional aspects"
\section{Non-functional aspects}
\label{sec:comparativenonfunctional}

% end of section comparativenonfunctional


% Section "Performance and resource consumption"
\section{Performance and resource consumption}
\label{sec:comparativeperformance}

% end of section comparativeperformance


% end of chapter
 % Chapter 5
  % !TEX root = ../Thesis.tex
% !TEX spellcheck = en-US

\chapter{Conclusions and further work}
\label{ch:conclusions}

In this chapter, our provisional conclusions, we're not doing nothing else than using a term in the glossary to force it to be printed.
We have decided that term to be~\gls{Real numbers}.

With this, we conclude this long document.
Thank you for reading!

% \section{Section 1}
% \label{sec:relsec1}

% \section{Section 2}
% \label{sec:relsec2}

% end of chapter
 % Chapter 6

% This ensures that the subsequent sections are being included as root
% items in the bookmark structure of your PDF reader.
\bookmarksetup{startatroot}
\backmatter

  \begingroup
    \let\clearpage\relax
    \glsaddall
    \printglossary[type=\acronymtype]
    \newpage
    \printglossary
  \endgroup

  \printindex
  \printbibliography

\end{document}
