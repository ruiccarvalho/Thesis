% !TEX root = ../Thesis.tex
% !TEX spellcheck = en-US

\chapter{Introduction}
\label{ch:introduction}

This first chapter introduces the challenge of using virtual, fully software-based alternatives to traditional, hardware-based computer network laboratories in education, with special focus on the undergraduate and graduate levels of university curricula.
It also explains the rationale for their use, and provides an overview of the work that has already been done both in development of this kind of tools and their current introduction, and study thereof, in universities' courses around the globe.
Furthermore, it contains the motivation behind the most important work in this dissertation, which consists of studying and comparing different software solutions for simulating and emulating computer networks and, both theoretically and practically, justifying how each one can, or cannot, be a good fit for particular use-cases in the study of computer networks.

\section{Motivation and Problem Statement}
\label{sec:motivation}

Teaching of computer networks courses and subjects, to fulfill the desired properties of a ``hands-on approach'' in engineering---one which enables putting in practice the knowledge acquired in the theoretical component of different subjects---typically requires performing two kinds of exercises~\cite{problembasedlearning}:
  \begin{enumerate*}[label=(\roman*), itemjoin={{, }}, itemjoin*={{, and }}]
  \item at the higher layers---application and transport---, using \glspl{api} that give access to the operating system's (OS) network stack and host interfaces
  \item at the lower layers of the stack, using physical laboratories, equipped with networking gear, i.e. switches and routers.
  \end{enumerate*}
The existence of such laboratories is, even today, standard practice in many institutions.
The Informatics Department (DI) of FCT/NOVA is not an exception, and has a classroom with routers and switches which students use to perform those practical exercises.

Despite its virtues, the use of network gear and interacting with real switches and routers (and their protocols) to learn the impact of different parameter values and of other nodes on the behavior of a complex application that uses the network raises problems such as the cost and fast obsolescence~\cite{automaticnetconfiggns} of hardware, and requires the students' presence in the laboratory (at least for manipulating physical links and network interfaces).
Other problems include possible damage due to misuse~\cite{teachinginovation} or simple wear-and-tear, and time and effort to prepare and setup those exercises, and that may ``destroy'' the setup for other, different exercises, etc.

Furthermore, overcoming the necessity of depending on specific geographic sites, of which classrooms in the university campuses are an example, is getting more and more desirable.
In fact, during the final stages of the project culminating in this document, the outbreak---and subsequent pandemic declared by the World Health Organization---of the disease caused by the SARS-CoV-2, COVID-19, has forced schools and universities to close and students, and citizens in general, to remain at home~\cite{covid19}.
In such a scenario, the importance of allowing students to be able to, at a distance, develop practical laboratories to follow along theoretical lectures---which keep happening thanks to videoconference and other solutions---is paramount.

\section{Contributions and Goals of this Dissertation}

Aiming to address these problems, which are not specific to higher education institutions, but shared with vocational education on high schools, industry courses (of which Cisco's CCNA~\cite{ccna} is a prominent example), and distance learning~\cite{networkvirtwithgns} institutions, tools for emulating and/or simulating networks and networking equipment have been developed for years.
These tools can, through the usage of several techniques, surpass some (if not all) of the aforementioned disadvantages.
Critically evaluate whether one or more, used in a complementary fashion, of the existing simulators/emulators can live up to that ``promise'' is, in a few words, the goal of this work.

Another contribution offered by this work is an in-depth study of the technologies used by the selected emulators, and the clarification of the reasons behind their behavior, performance-wise.

\section{Structure of the Document}
\label{sec:structure}

The content of the dissertation is organized in the following way:
\begin{itemize}
  \item \textbf{Chapter~\ref{ch:leavingthephysicalworld}} differentiates emulators from simulators and introduces fundamental concepts and techniques of software-based solutions that aim to replace hardware based ones, with a focus in the applications of interest for the goal of this dissertation;
  \item \textbf{Chapter~\ref{ch:examplesofemulators}} gives an overview of a selected set of emulators chosen to exemplify different approaches to the problem of network emulation;
  \item \textbf{Chapter~\ref{ch:gns3}} describes the GNS3 emulator in terms of its capabilities, technical underpinnings, and how it could be used to implement some lab exercises commonly found in computer network curricula;
  \item \textbf{Chapter~\ref{ch:kathara}} describes Kathará, an emulator substantially different from GNS3. The description is presented along the same lines as before, i.e. when describing GNS3;
  \item \textbf{Chapter~\ref{ch:core}} describes CORE, yet another emulator that, whilst different from the two above, shares common traits with both;
  \item \textbf{Chapter~\ref{ch:comparative}} presents, in a way as systematic as possible, the pros and cons of the emulators that we have studied and tested in the three previous chapters;
  \item \textbf{Chapter~\ref{ch:conclusion}} presents conclusions and proposes work that can be done in the future in terms of how the existing emulators and their underlying technologies can be better used;
  \item \textbf{Appendix~\ref{ch:listings}} contains selected examples of code and text-based configuration data files.
\end{itemize}

% end of chapter
