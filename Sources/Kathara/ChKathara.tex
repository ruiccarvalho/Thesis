% !TEX root = ../Thesis.tex
% !TEX spellcheck = en-US

\chapter{Kathará}
\label{ch:kathara}

Kathará~\cite{kathara} is a new implementation, created at the Roma Tre University using more modern technologies, of Netkit, a network emulator developed at the institution years before.
It was made with the purpose of introducing functionalities of emulating SDN and NFV based networks, as well as leveraging the P4 language for programming forwarding-planes.
Still, Kathará---as explicitly stated in the previously cited presentation article---is fully compatible with Netkit, having a similar interface and conceptual philosophy than its predecessor, in particular fully supporting topologies based in ``standard'' routing devices and generic hosts.

\section{What is it}
\label{sec:katharawhatis}

\section{Architecture}
\label{sec:katharaarchitecture}

\section{GUI to setup projects}
\label{sec:katharagui}

\section{Setup of the topology and results}
\label{sec:katharatopologyexample}

% end of chapter
