% !TEX root = ../Thesis.tex
% !TEX spellcheck = en-US

\chapter{Comparative Analysis}
\label{ch:comparative}

In this chapter, the results of the experimentation with and in-depth study of GNS3, Kathará, and CORE (chapters~\ref{ch:gns3}, \ref{ch:kathara}, and~\ref{ch:core}, respectively) are exposed in a comparative way, so that the strengths and weaknesses of each one---either in general or from an ``absolute'' point of view, but especially for relative to the application intended in this dissertation, which is teaching, learning, and studying in the context of university courses---are clearer and an informed choice can be made, for example, in the moment of planning a course.

The structure of the chapter is as follows:
\begin{itemize}
  \item In section~\ref{sec:comparativefunctionality}, we compare the functional aspects of the two emulators, i.e. what \emph{can} they do and what they \emph{can't}.
  \item In section~\ref{sec:comparativenonfunctional}, we focus on most of the non-functional aspects, namely how user friendly they are to perform some of the tasks and how some design decisions make each one more suitable for some types of usage than for others.
  \item Section~\ref{sec:comparativeperformance} is dedicated to a particular non-functional attribute of the emulators, which is performance (and resource consumption).
\end{itemize}

% Section "Functionality"
\section{Functionality}
\label{sec:comparativefunctionality}

Functionality, in particular in the emulated network-nodes, is defined, in this document, in two senses:
\begin{enumerate}
  \item From a \emph{higher-level} perspective, the possibility to run \emph{generically} existing algorithms and protocols, exchanging real packets in the respective layers according to the standard way to do so.
  \item From a \emph{lower-level} point of view, the ability to run vendor-specific software, with all of its particular attributes, maybe proprietary optimizations, augmented headers, or any kind of functionality that a particular ``brand'' of networking software may offer.
\end{enumerate}

\subsection{Higher-level Functionality}

In terms of the \emph{higher-level functionality}, most general-purpose emulators, which GNS3, Kathará, and CORE are at heart, don't differ much.
They all have a way to define arbitrary topologies and store them in the filesystem and the topologies describe nodes, default configuration of the nodes, and hosts represent ``computers'' that, according to the specifics of the software/firmware they are running and number of interfaces serve as switches, routers, end-hosts running application software (client/server, P2P, etc.), or even other kinds of nodes seen in real-world networks, often called middleboxes, not studied in the present work.

An important limitation with Kathará and CORE has to be noted, though.
Among all the documented labs, none is related to bridging/layer-2 switching.
With Netkit, Kathará's precursor, whose implementation technologies were different, that wasn't the case---and exercises to test the working of STP in a switched network were documented.
In fact, if we try to mimic those labs, as well as APRC's lab2-handout (cf.~\ref{sec:katharapracticalcasestudy}), they don't work---despite capturing network traffic between nodes shows that STP is communicating, the Linux bridges.
This was not tested with CORE.

Thus, as far as our experiments go, for layer-3 switching (data-plane) and routing (control-plane), every APRC exercise can be implemented in any of the three emulators.

According to our tests, the layer-2 exercises are not possible in Kathará.
They were not tested in CORE.

\subsection{Lower-level Functionality}

If, for example, running Cisco software is a requirement (as is the case for the company's official certifications), Kathará or CORE themselves cannot accommodate it.
On the contrary, there aren't officially supported GNS3 appliances, running on any kind of platform, to run Quagga out of the box.
This is to show that, \textbf{in an academic context}, where protocols and algorithms are the subject of the study, and not vendor-specific details, \textbf{Kathará and CORE have the advantage of doing without pricey licenses and vendor lock-in}.

The degree of flexibility of the two emulators is different, due to reasons probably obvious by now.

On the one hand, GNS3 is ``a suite of \emph{emulation methods}'' (and even more than that, giving the diversity of software packages that are inside a normal installation of the software on a desktop), and offers the ability to emulate nodes using large span of technologies that don't typically work with each other by making ensuring that one single software package, uBridge, is pluggable to each of them, being responsible to handling the transmission of the traffic, in a way that each node's emulation method is never aware of the emulation method on the other end of a virtual link.

On the other hand, Kathará and CORE do not use any application-level data-link emulation, and require that every kind of software, be it for switching/routing nodes, middleboxes, is able to communicate using Docker's virtual interfaces (Kathará) or the bridges between the network spaces' \glspl{nic}.
Kathará, CORE, and GNS3 provide a way to connect an interface on the host operating system to a virtual interface on its nodes.

This, if thought in conjunction with the way hypervisors like VMware Workstation allow to create virtual switches and interfaces in the host computer, allows for virtually unlimited combination of topologies across both emulators, using physical interfaces, and even mixing them with other ones on the Internet.

% end of section comparativefunctionality


% Section "Non-functional aspects"
\section{Non-functional aspects}
\label{sec:comparativenonfunctional}

% end of section comparativenonfunctional


% Section "Performance and resource consumption"
\section{Performance and Resource Consumption}
\label{sec:comparativeperformance}

In terms of resource consumption, experiments with GNS3, on one side, and Kathará and CORE, on the other, are very different.
That has an impact in the required computer infrastructures and setups required to do the experiments.
As stated in the explanation between containers-as-lightweight-virtual-machines versus standard \glspl{vm}, a container provides a virtual isolation of the filesystem and networking (among others) for its processes while \glspl{vm} need everything, in every layer of software, a regular host has to run (operating system, libraries, applications) to be loaded each time for each running virtual machine.

In a running Kathará, CORE, or GNS3 topology, there are associated processes to each node.
As previously shown, GNS3's Cisco routers can be \glspl{vm} running the modern Cisco IOSv or processes of the Dynamips emulator which in fact is a virtual machine, but not in the sense of \textquote{a PC running in isolation with virtualized IO} (e.g. the instructions of those IOS images isn't x86, and therefore there has to be a translation to machine instructions).

Table~\ref{tab:comparativeramusage} lists approximate values measured with the \texttt{top} command (and \texttt{docker~stats}, for Kathará's) on more than one GNU/Linux system for the memory consumption of each process that backs a live node in a topology for each studied possibility.
These measures only approximate the order of magnitude, in a sense that can be described as: \textquote{of several different times the values were measured, on more than one machine, they didn't fluctuate more than, say, $50~\mbox{MiB}$ (for GNS3's), $5~\mbox{MiB}$ (for Kathará's), and $1~\mbox{MiB}$ (for CORE's)} and therefore it doesn't correspond to an accurate arithmetic mean or other formal statistical method.

% Table tab:comparativeramusage
\begin{table}
  \centering
  \small
  \begin{tabulary}{0.9\textwidth}{ll}
    \toprule
      \textbf{Topology node}                   & \textbf{Memory usage}\\
    \midrule
      GNS3 -- Cisco IOSvL2 (KVM via QEMU)      & $\approx 5~\mbox{MiB}$\\
      GNS3 -- Cisco c3745 (Dynamips process)   & $\approx 270~\mbox{MiB}$\\
      Kathará (any node is a Docker container) & $\approx 460~\mbox{MiB}$\\
    \bottomrule
  \end{tabulary}
  \caption{%
    Approximate metrics of the memory footprint of (networking) nodes in topologies for different emulators
  }
  \label{tab:comparativeramusage}
\end{table}


For the sake of example, the query for the memory footprint of the running Docker containers on a host (in this case, all of them are Quagga-enabled Linux hosts as Kathará nodes) is shown in figure~\ref{fig:comparative-docker-stats}.

CORE's measure of (idle) memory footprint for a sample router was made summing the memory footprint of an instance of the container launcher process (\texttt{vnoded}), \texttt{ospf6d}, \texttt{ospf}, and \texttt{zebra}.

% Figure fig:comparative-docker-stats
\begin{figure}
  \centering
  \includegraphics[width=0.8\textwidth]{comparative-docker-stats}
  \caption{The \texttt{docker stats} command showing the resources taken by containers corresponding to nodes in a Kathará lab}
  \label{fig:comparative-docker-stats}
\end{figure}


% end of section comparativeperformance


% end of chapter
