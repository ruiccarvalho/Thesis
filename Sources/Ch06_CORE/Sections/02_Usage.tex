\section{Usage}
\label{sec:coreusage}

The standard usage of CORE is through its \gls{gui}.
It requires the daemon to be on the host-OS and is executed as user-level application.

With the \gls{gui}, the user can edit a special kind of text file with the nodes, their \emph{types}, the links between them, default IP addresses for the interfaces, and the desired \emph{services}.

A designed topology can be started, which will make the CORE daemon to create all the containers and start the pre-selected services with the default parameters.
Then, the terminal windows can be launched through the user interface to issue shell commands to each container.
The available commands, and software in general, are the ones available in the host OS.
If, for instance, \texttt{iperf3} is to be used, it has to be installed globally on the host running the emulation session.

\emph{Services} are the entities consisting of software set-up in a ``plug-in'' fashion---either provided by default by CORE, or user made, according to the steps described in the documentation---, that can be easily enabled, disabled, started, and stopped for all or some nodes.
The Quagga routing suite (with its many daemons for different routing protocols) is a service. Apache's \texttt{httpd} is also a service.

Node \emph{types} are the set ``templates'' of different kinds of emulated machines that can be added to a canvas. They range from routers or switches to a simple PC or a firewall.
These types differ on the services that come enabled on them by default, and also on the parameters that can be set through menu-dialogs which are automatically translated to internal configurations, depending of the context.
A node of type \emph{host} has an SSH service running, while a node of the \emph{router} kind has the Quagga service with Zebra and OSPF daemons enabled by default.
The full list can be found at~\cite{coredocs}.

% end of section
