% !TEX root = ../Thesis.tex
% !TEX spellcheck = en-US

\cleardoublepage\thispagestyle{plain}

\begin{otherlanguage}{portuguese}
  \textbf{\Large Resumo}

  Nos últimos anos têm sido desenvolvidos emuladores de redes, baseados sobretudo em tecnologias de virtualização, que permitem executar experiências que reproduzem de forma bastante realista o que se passa numa rede real.

  Paralelamente, não só os computadores pessoais (PCs) têm hoje muito mais recursos (dimensão da memória, velocidade do processador, etc.) do que há alguns anos, como a conectividade à Internet é cada vez mais omnipresente, facilitando o contacto com infra-estruturas poderosas, na chamada \emph{cloud} ou em centros de dados institucionais, com os quais se pode contar para montar experiências com emuladores, nos casos em que estas requeiram mais recursos do que os disponíveis num PC.

  Esta tese avalia crítica e comparativamente a forma como dois exemplos paradigmáticos de emuladores modernos, o GNS3 e o Kathará, permitem dar resposta a problemas existentes no ensino, aprendizagem e investigação experimental em redes de computadores, nomeadamente se é possível, graças a eles, mitigar ou eliminar a dependência de um laboratório com equipamento físico para a realização de certos exercícios práticos.

  \textbf{Palavras-chave:} Redes de computadores; emuladores; virtualização; educação.
\end{otherlanguage}
