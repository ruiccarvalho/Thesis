% !TEX root = ../Thesis.tex
% !TEX spellcheck = en-US

\cleardoublepage\thispagestyle{plain}

\begin{otherlanguage}{portuguese}
  \textbf{\Large Resumo}

  Nos últimos anos têm sido desenvolvidos emuladores de redes, baseados sobretudo em tecnologias de virtualização, que permitem executar testes que reproduzem de forma bastante realista o que se passa numa rede real.

  Paralelamente, não só os computadores pessoais têm hoje muito mais recursos (memória, processador, etc.) do que há alguns anos, como a conectividade à Internet é cada vez mais omnipresente, facilitando o contacto com infra-estruturas, na chamada \emph{cloud} ou em centros de dados institucionais, com os quais se pode contar para montar experiências com emuladores.

  Esta tese avalia crítica e comparativamente como é que dois exemplos paradigmáticos, o GNS3 e o Kathará, permitem dar resposta a problemas existentes no ensino, aprendizagem e investigação experimental em redes de computadores, nomeadamente se permitem diminuir a dependência de um laboratório com equipamento físico para a realização de certos exercícios práticos.

  \textbf{Palavras-chave:} Redes de computadores; emuladores; educação
\end{otherlanguage}
