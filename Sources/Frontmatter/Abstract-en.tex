% !TEX root = ../Thesis.tex
% !TEX spellcheck = en-US

\cleardoublepage\thispagestyle{plain}

\textbf{\Large Abstract}

In recent years computer networks emulators have been developed, mostly relying on virtualization technologies, which allow performing experiments that reproduce in a quite realistic way what happens in a real network.

In parallel, not only do personal computers (PCs) have nowadays much more resources (memory capacity, processor speed, etc.) than a few years ago, but Internet connectivity is increasingly ubiquitous, facilitating contact with powerful infrastructures, in the so-called \emph{cloud} or in institutional data-centers, which can be relied upon to set up experiments with emulators, in cases where these require more resources than those available on a PC.

This thesis critically and comparatively assesses how two paradigmatic examples of modern emulators, GNS3 and Kathará, provide answers to existing problems in teaching, learning, and experimental research in computer networks, namely whether it is possible, thanks to them, to mitigate or eliminate the dependency on a laboratory with physical equipment to perform certain practical exercises.

\textbf{Keywords:} Computer networks; emulators; virtualization; education.
