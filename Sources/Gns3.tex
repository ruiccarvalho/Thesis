% !TEX root = ../Thesis.tex
% !TEX spellcheck = en-US

\chapter{GNS3}
\label{ch:gns3}

GNS3, whose original title was ``Graphical Network Simulator-3,'' is a software project, comprising several distinct components and, despite the ``simulator'' in its name, \emph{as a whole} falls under the category that, in this dissertation, we call an \emph{emulator}.
By ``as a whole'', it is meant that, as shall be seen later, although some of its components, like the Dynamips program, are emulators in a strict sense---i.e. serve to run real machine code on a different (than its native one) hardware architecture---, it differenciates itself on a high-level perspective from a simulator which is a program designded to execute a mathematical model, processing modeled events as internal data-structures with a set of preset algorithms that somehow mimic a (part) of the reality.

% end of intro

\section{GNS3's purpose and \emph{raison d'être}}
\label{sec:gns3why}

% end of section gns3why

\section{Building blocks. The programs inside ``GNS3''}
\label{sec:gns3buildingblocks}

% end of section gns3buildingblocks

\section{General architecture}
\label{sec:gns3architecture}

% end of section gns3architecture

\section{GNS3 in action}
\label{sec:gns3inaction}

% end of section gns3inaction

\section{Performance and resources considerations}
\label{sec:gns3performance}

% end of section gns3performance

% end of chapter
