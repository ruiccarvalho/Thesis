\section{GNS3's purpose and \emph{raison d'être}}
\label{sec:gns3why}

The GNS3 project was co-created by Jeremy Grossmann at the University, the Ecole informatique Epitech, as part of the \emph{EPITECH Innovative Project (EIP)}\footnote{\url{https://eip.epitech.eu/2013/gns3/en/index.html}, accessed on December 2019}.
There is also a blog\footnote{\url{http://gns3.blogspot.com/2007/}, accessed on December 2019}, whose first posts date back to 2007, that announces the first ``beta'' releases of the software and gives the details about the inception of GNS3, and also lists the original developers of the project.

It may be speculated that GNS3's first ``appeal'' was supplying students of Cisco certifications with a self-study tool more powerful that anything before, one that allows for the creation of arbitrary network topologies and practice their skills with the same software stack that is used in real Cisco devices and the hosts connected to them---from the operating system, up until application--level network utilities---, without having to use the expensive official solutions for that purpose, or depending on a real, physical networking laboratory.
However, despite having kept a strong relation with training for Cisco CCNA and related certifications~\cite{thebookofgns3,gns3netsimguide}, the magnitude of the project, part of which comes from an inherent extensibility, has made it suitable for a myriad of use-cases.

In the industry, GNS3 naturally can come handy as a tool to prototype and test a topology for a real organization network, in the context of the practice of a network professional.
But also, in the age of the DevOps philosophy---a set of practices and methodologies/philosophies for faster delivery of applications and services---, of which \emph{infrastructure as code} is one of the culprits~\cite{awswhatisdevops}, it can be of value for software engineers to test distributed applications that depend on certain network behaviors.

% At the moment of the writing of this thesis, version 2.2 has recently came out\footnote{\url{https://github.com/GNS3/gns3-gui/releases/tag/v2.2.0}}.
% This version brings a lot of improvements and new features, like the new web UI or ``physical'' link state detection for QEMU nodes~\cite{releasenotesgns3v22}, and at the same time some of the ways GNS3 has been typically used are changing---such is the case of the discouragement of the usage of Dynamips and VPCS (in detriment of virtualized versions of IOS or others, or Docker containers for ``emulating'' hosts, respectively)~\cite{ytdynamipsvpcs}.
