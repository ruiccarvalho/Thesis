% !TEX root = ../Thesis.tex
% !TEX spellcheck = en-US

\chapter{Conclusion}
\label{ch:conclusion}

This chapter offers a summary of the work done throughout this dissertation.
It then goes on to highlight ideas that can possibly be useful to explore in the future.

\section{Summary}

This dissertation approached the problem of evaluating the suitability of emulators for its use in teaching and learning---how they work, what's necessary to use them, what are their limitations\textellipsis

This work provided a clearer picture of the current situation of emulators.
The technical underpinnings of most of them, which is strictly related to how well they perform in certain cases, and also their weaknesses.
It also clearly shows in what aspects a real course---in particular the one chosen for the case studies, APRC---can benefit from any of the studied emulators.

In short:
\begin{itemize}
  \item we saw that emulators bring clear benefits over simulators, e.g. that it is possible to use real-world tooling to perform experiments on top of them and see how real routing software, network applications, and operating systems behave in different topologies and under certain parameters;
  \item we presented a survey of different emulator software with a high-level perspective of their claimed strengths and weaknesses and main usage targets;
  \item out of the surveyed emulators, we studied three of them in detail and took conclusions out of the study of their technical architecture and implementation techniques, and usability aspects;
  \item we performed a side-by-side comparison of these emulators concerning a fixed set of functional and non-functional metrics.
\end{itemize}

The conducted experiments, the knowledge gained about the internal properties of the studied emulators, and the documented usages already in place in certain areas of the academia and the industry clearly point towards the conclusion that many more experiments than the ones described in this dissertation are possible and that, if not completely (for reasons already explained in this document), emulators can live up to the promise of replacing physical laboratories, equipped with network gear, to large extent.

\section{Future Work}

Despite all that was done, there are many more aspects of computer networks studying that were not approached here---some of them are part of the chosen course's curricula (like Open vSwitch and \gls{sdn}), other are not, like programmable routers with the recent P4 network or NFV.
Seeing how to profit from emulators to introduce new practical topics.

Also, a more fine-grained testing can be done, both in terms of the how these platforms work under scale (more nodes, more traffic between the nodes) but also being used by many students at the time.

Testing the pedagogical results and the methodological fine-tuning necessary in a networks course (say, like APRC) to work exclusively in a software based solution, and measure the degree of success and failure of such an approach is also essential.

Finally, two use-cases were left out of this work that are of interest:
\begin{itemize}
  \item using emulators for application-level exercises (like the ones performed in the FCT/NOVA's Informatics Engineering undergraduate Computer Networks course) and benefit from a reproducible and well-controlled environment;
  \item programming exercises consisting of implementing routing algorithms and protocols, of which the work done in~\cite{teachandlearnmininet} is an example.
\end{itemize}

% end of chapter
