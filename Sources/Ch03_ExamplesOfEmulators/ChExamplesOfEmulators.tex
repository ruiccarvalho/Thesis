% !TEX root = ../Thesis.tex
% !TEX spellcheck = en-US

\chapter{Examples of emulators}
\label{ch:examplesofemulators}

This chapter offers an overview of four emulators, all different from one another in implementation, functionality, extensibility, user interaction capabilities, and even in the problems they were built to address.

It is not an in-depth approach, but an introduction to allow to establish what we believe is a representative sample of approaches in the field, trying to describe the context of their conceptions, as well as details relevant to highlight the differences between them in the aforementioned aspects.

% Section "Mininet"
\section{Mininet}
\label{sec:exemulmininet}

Mininet~\cite{mininetnetworklaptop} is a network emulator created by researchers at Stanford University, intended to be lightweight, and allow, as the title of the paper that presented it for the first time says, \textquote{rapid prototyping for Software-Defined Networks.}

Like essentially every emulator able to run on commodity PCs, standalone, Mininet makes use of virtualization.
It was created specifically with that in mind, as an alternative to the limited simulators, but also expensive testbeds, that require a layer of real software, even if provided ``as a service'' to the users.
It also has had from the beginning heavy concerns about performance and the ability to run large topologies on the researchers' machine.

Mininet operates exclusively on Linux hosts and does everything with the low-level primitives of LXC (Linux Containers).
This enabled very large topologies to be run without much effort from the host they're supported by.

The kinds of nodes Mininet supports can't be dissociated from the fact that is was developed with SDN in mind.
They are:
\begin{description}
	\item[hosts] corresponding to network namespaces, containers, consisting in processes with exclusive ownership of interfaces, ports, and routing tables;
	\item[switches] these are OpenFlow ``dumb'' switches, \textbf{not} ``conventional'' layer-2 switches, which can run in kernel-space or user-space and be configured by the Mininet infrastructure;
	\item[controllers] which can be on the ``virtual'' topology, but also be proper nodes in a deployed SDN, as long as the host where the other nodes are running has IP connectivity to the rest of the network.
\end{description}

There is a command-line utility to interact with Mininet, which serves to add and remove, and also start and stop, topology nodes, access their shells.
It is also possible to automate these actions and declare shareable topologies using Mininet's Python API.

It is out of the scope of this thesis studying in details what can be done with the SDN approach e.g. in terms of programming the control plane of switches, as well as how that is done, since the focus from the beginning is to reproduce experiments currently runnable with standard, layer-2 switches and conventional IP routers.

To respond goals of the present work, even though the generality suggested by the simple definition of Mininet's \emph{host} nodes may lead one to assume that they could use something like Quagga~\cite{quagga} and an arbitrary number of interfaces to emulate a multilayer switch, both the lack of proper authoritative documentation to do so, as well as the results of Web searching, suggest that, for various reasons that is not easy or even feasible, at least maintaining two of the claimed ``good'' attributes of Mininet: deployability and sharability.

Yet another reason for not investing in struggling against the ``stock'' limitation of Mininet is the fact that, as will be seen in other examples, there are other solutions, some able to run in more platforms (any host/OS supporting Docker out-of-the-box), which provide good, documented support for conventional open-source switches/routers (e.g. Quagga), maintaining the positive aspects of Mininet like declarative, scriptable topologies, and lightweight virtualization for large topologies.

% end of section exemulmininet


% Section "GNS3"
\section{GNS3}
\label{sec:exemulgns3}

GNS3, whose original name was ``Graphical Network Simulator-3,'' is a software project, comprising several distinct components and, despite the ``simulator'' in its name, it falls under the category of emulators (cf.~section~\ref{sec:leavingemulationandsimulation}).

The GNS3 project was co-created by Jeremy Grossmann as a University project, at the Ecole informatique Epitech, part of the \emph{EPITECH Innovative Project (EIP)}.\footnote{\url{https://eip.epitech.eu/2013/gns3/en/index.html}, accessed on December 2019}
There is also a blog,\footnote{\url{http://gns3.blogspot.com/2007/}, accessed on December 2019} whose first posts date back to 2007, that announces the first ``beta'' releases of the software and gives the details about the inception of GNS3, and also lists the original developers of the project.

% It may be speculated that GNS3's first ``appeal'' was supplying students of Cisco certifications with a self-study tool more powerful that anything before, one that allows for the creation of arbitrary network topologies and practice their skills with the same software stack that is used in real Cisco devices and the hosts connected to them---from the operating system, up until application--level network utilities---, without having to use the expensive official solutions for that purpose, or depending on a real, physical networking laboratory.
Unlike Mininet, CORE, and Kathará, GNS3 wasn't created with ``lightweigthness'' in mind, but with the goal of providing as much as possible the ability to create rich and heterogeneous topologies with real, commercial router software (such as Cisco IOS) running in the switching and routing nodes.
But also ready to add \glspl{nat} and firewalls, as well as full-blown virtual machines serving as hosts.
All of this because GNS3's strong major strength is high functional realism and compliance with industrial production environments, making it a natural choice for the self-study of students  training for Cisco CCNA and related certifications~\cite{thebookofgns3,gns3netsimguide}, as well as a clever architecture for load distribution that makes it relatively simple to scale high-resource demanding large and complex topologies and experiments.

However, the magnitude of the project, part of which comes from an inherent extensibility, has made it suitable for a myriad of use-cases.

In the industry, GNS3 naturally can come handy as a tool to prototype and test a topology for a real organization network, in the context of the practice of a network professional.
But also, in the age of the DevOps philosophy---a set of practices and methodologies/philosophies for faster delivery of applications and services---, of which \emph{infrastructure as code} is one of the culprits~\cite{awswhatisdevops}, it can be of value for software engineers to test distributed applications that depend on certain network behaviors.

% \subsubsection{Usage of GNS3 in education}
% \label{subsubsec:relgns3usageedu}

There is already some published work about using GNS3 in education institutions.
Usually, though, the experiments and comparisons are made against other ``graphical simulators''---instead of emulators---, since those are the software tools that, from a high-level perspective are equivalent, even if the functional possibilities and implementations are completely different, if not opposite.

A brief comparison with Boson NetSim and Packet Tracer, two graphical simulators, is presented in~\cite{virtlabgnsvmware}, where the authors explain a way to use VMware Workstation \glspl{vm} on a desktop to ``emulate'' end hosts on GNS3's virtual topologies, and using applications on those \glspl{vm} which communicate with each other over the emulated links.

In~\cite{automaticnetconfiggns}, the authors, two Professors from the Polytechnic Institute of Leiria, Portugal, dive into ways to automate GNS3 topology configurations, which, as will be seen in chapter~\ref{ch:gns3}, can be a major issue. % TODO

The usefulness for distance learning is also explored, for example, in~\cite{networkvirtwithgns}.

The presentation of GNS3 made here is incomplete regarding both technical aspects and also its functionality and mode of operation.
This is due to the fact that this tool will be examined in much more detail later on.

% end of section exemulgns3


% Section "Kathará"
\section{Kathará}
\label{sec:exemulkathara}

Kathará, presented as \textquote{Container-Based Framework for Implementing Network Function Virtualization and Software Defined Networks,} is an emulation framework developed at the Roma Tre University~\cite{kathara}.
It was created with the purpose of building a tool that, simultaneously, allowed to prototype and research with concepts considered ``recent'' or trending in networking and related fields.

According to its developers, Kathará was built having in mind the creation of software framework, a computer network emulator, that allowed to prototype, develop over, and experiment with, not only traditional switched and routed networks, but also with four emerging concepts and technologies in the scope of (or related to) networking:
\begin{itemize}
	\item \gls{sdn}, enabling OpenFlow switches and their corresponding controllers to be added to topologies, allowing for separate data-plane-control-plane devices;
	\item \gls{nfv}, supporting ClickOS~\cite{clickos} (a software system to provide network functions on lightweight middleboxes) nodes which can provide \glspl{vnf} such as \glspl{nat}, firewalls or \glspl{dmz};
	\item P4~\cite{p4programming}, a domain specific language that is compiled and run on switching/routing devices providing the concept of a ``programmable data-plane'' (in the node, instead of separately, as in the ``original'' \glspl{sdn});
	\item containers (and, in particular, Docker) as the \textquote{lightweight and multi-platform alternative to standard virtualization solutions \ldots\ an isolated environment in which processes run independently, avoiding the overhead introduced by full virtualization that needs to maintain multiple kernels.}
\end{itemize}

However, despite this motivation, Kathará implements the same topology definition syntax and user interface that Netkit~\cite{netkit-short} has, an emulator developed years before at the same institution, does, and supports, from a high-level perspective, the same capabilities for experimenting with traditional networks comprising distributed routing devices.
Like with the previous example, since this is one of the emulators being studied in more detail in the present work, no technical underpinnings are being presented in this section, only high-level features.

Unlike GNS3, Kathará doesn't follow a ``graphical first'' approach to its UI, since it doesn't offer, out-of-the-box, any visual topology editor nor allows to manipulate running labs via graphical controls.
Instead, it privileges a declarative approach for the definition of the labs/topologies as well as the usage of a command-line interface to start, stop, and interact with nodes.
In this sense, it is relatively similar to Mininet, but it leverages the decision of using Docker containers to provide a very ``extensible-by-design'' functional philosophy, instead of being geared towards a strict \gls{sdn}-related mindset.
Kathará explicitly wants to provide a rich environment to combine the emulation of latest software-defined and highly virtualized networks with the ``traditional'' model that Netkit already used to provide.

Kathará is essentially an academic tool.
As such, it is not oriented towards industry or vendor specific details, offering its exclusively through free and/or (partially) open-source software.

% end of section exemulkathara


% Section "CORE"
\section{CORE}
\label{sec:exemulcore}

The CORE~\cite{coreemulator} emulator was developed originally by Jeff Ahrenholz at the Boeing Company.
Originally, CORE was an acronym meaning Common Open Research Emulator.
Its official website\footnote{\url{https://www.nrl.navy.mil/itd/ncs/products/core}} is hosted under the Networks and Communication Systems Branch of the U.S. Naval Research Laboratory, the institution currently responsible by its maintenance~\cite{Peach2016AnOO}.

It is a fork of a pre-existing project, Integrated Multiprotocol Network Emulator/Simulator (IMUNES), from the University of Zagreb.

It's described as \textquote{a tool for building virtual networks. As an emulator, CORE builds a representation of a real computer network that runs in real time, as opposed to simulation, where abstract models are used.}
Networks emulated with it can be, via the host machine of the emulation, connected to real world, physical networks.
Its documentation~\cite{coreghdocs} also states that \textquote{it provides an environment for running real applications and protocols, taking advantage of tools provided by the Linux operating system.}

Like the other emulators studied in this dissertation, except for GNS3, CORE is a framework for leveraging very lightweight OS-virtualized hosts, also known as containers, as a means to run isolated network stacks for each node---be it routers/switches, hosts, or any network service that can be run on top of Linux.

It is particularly similar to Kathará in that it offers a very wide range of services (essentally, appliances) working out of the box right after being added to a topology, relying on specific combinations of open-source software running on each node.

On the other hand, like GNS3, it provides a \gls{gui} as the default fashion to edit topologies and control a running emulation work session in real time, rather than a pure textual/declarative language to specify and configure projects (topologies) and a command-line interface to control the emulation~\cite{coreghdocs}.

A particular trait of CORE which distinguishes it from the other simulators is its readily available support a special kind of wireless networks emulation/simulation, which may take into consideration, for routing purposes, the geometrical position of the nodes in the topology canvas and, according to a user-set scale, translates it into a geographical disposition, which can in turn produce different results~\cite{ospfmanet}.

% The first versions of CORE, like its predecessor IMUNES, were based upon both FreeBSD' networking stack and ``jails'' (a particular implementation of containers by that BSD operating system~\cite{freebsdjails})~\cite{comparisonofcore}.

% Nowadays, though CORE relies on one of the ways provided by the Linux kernel to enable ``containerization,'' which, as seen earlier in this text, is a way to isolate, at some degree, processes from each other, in this case with special attention to the IO and in particular the OS's networking stack. % TODO acronym for OS

% CORE comprises multiple parts, but from a high-level standpoint what matters the most is that it has offers a \gls{gui} as the main, official way to interact with the system, and then has a daemon running as a service in the host's Linux setup that is its backend which ensures the creation of namespaces and orchestrates the virtual bridges between them (which support the emulated links).

% It is documented to be able to run in a distributed fashion, like GNS3 (cf. ~\ref{sec:exemulgns3}), so that large topologies with a lot of nodes can be put set in motion, distributing that load of virtual nodes across multiple hosts.

% The support for a regular layer-3 network with standard routers is provided by Quagga~\cite{quagga}.
% All nodes are abstracted as GNU/Linux hosts, which can have a series of \emph{services} running.
% The set of services running by default depends on the type (a PC, layer-3 router, layer-2 switch, etc.) that is selected for each node in a project, but can be further customized.
% The full list of the services available can be consulted in~\cite{coreemuservices}, but we highlight. % TODO preencher isto e pôr uma tabela

% end of section exemulcore


% Section "Summary"
\section{Summary}
\label{sec:exemulsummary}

In this chapter we gave an overview of several network emulators, namely Mininet, GNS3, Kathará (together with its predecessor, Netkit), and CORE.

In the next three chapters, we will dive deep in the analysis of functional and non-functional aspects, and also practical usage tests, for the last three of the aforementioned emulation tools.

% end of section exemulsummary


% end of chapter
