% !TEX root = ../Thesis.tex
% !TEX spellcheck = en-US

\chapter{Examples of emulators}
\label{ch:examplesofemulators}

This chapter offers an overview of four emulators, all different from one another in implementation, functionality, extensibility, user interaction capabilities, and even in the problems they were built to address.

It is not an in-depth approach, but an introduction to allow to establish what we believe is a representative sample of approaches in the field, trying to describe the context of their conceptions, as well as details relevant to highlight the differences between them in the aforementioned aspects.

% Section "Mininet"
\section{Mininet}
\label{sec:exemulmininet}

Mininet~\cite{mininetnetworklaptop} is a network emulator created by researchers at Stanford University, intended to be lightweght, and allow, as the title of its founding paper says, \textquote{rapid prototyping for Software-Defined Networks.}

% end of section exemulmininet


% Section "GNS3"
\section{GNS3}
\label{sec:exemulgns3}

GNS3, whose original name was ``Graphical Network Simulator-3,'' is a software project, comprising several distinct components which, despite the ``simulator'' in its name, falls under the category of emulators (cf.~section~\ref{sec:leavingemulationandsimulation}).

The GNS3 project was co-created by Jeremy Grossmann as a University project, at the École Informatique Epitech, part of the \emph{EPITECH Innovative Project (EIP)}.\footnote{\url{https://eip.epitech.eu/2013/gns3/en/index.html}, accessed on December 2019}
There is also a blog,\footnote{\url{http://gns3.blogspot.com/2007/}, accessed on December 2019} whose first posts date back to 2007, that announces the first ``beta'' releases of the software and gives details about the inception of GNS3, and lists its original developers.

Unlike Mininet, CORE, and Kathará, GNS3 was created with the goal of providing as much as possible the ability to create rich and heterogeneous topologies capable of running commercially available software (such as Cisco IOS), in GNS3 emulated switching and routing nodes, and was not particularly focused on being high performance and/or lightweight.
GNS3's list of capabilities also include the ability to run full-blown virtual machines, e.g. serving as hosts, and perform \gls{nat} and firewalling functions.

All of this is possible because GNS3's major strength is high functional realism and compliance with industrial production environments---making it a natural choice of students training for Cisco CCNA and related certifications~\cite{thebookofgns3,gns3netsimguide}---, as well as because of a clever architecture for load distribution that makes it relatively simple to scale high-resource demanding large and complex topologies and experiments.

The magnitude of the project, part of which comes from its inherent extensibility, has made it suitable for a myriad of use-cases.
For example, in the industry, GNS3 naturally can be a handy tool to prototype and test a topology for a real organization network, in the context of the practice of a network professional.
But also, in the age of the DevOps paradigm---a set of practices and methodologies for faster delivery of applications and services---, of which \emph{infrastructure as code} is one of the culprits~\cite{awswhatisdevops}, GNS3 can be of value for software engineers to test distributed applications that depend on certain network behaviors.

There is already some published work on the use GNS3 in education.
Usually, though, the experiments and comparisons are made against other ``graphical simulators''---instead of emulators---, since those are the software tools that, from a high-level perspective, are equivalent, even if the functional possibilities and implementations are completely different, if not opposite.

A brief comparison with Boson NetSim and Packet Tracer, two graphical simulators, is presented in~\cite{virtlabgnsvmware}---the authors explaining a way to use VMware Workstation on a desktop to deploy \glspl{vm} on GNS3's virtual topologies, acting as hosts, and run applications which communicate with each other over the emulated links.

In~\cite{automaticnetconfiggns}, the authors, two Professors from the Polytechnic Institute of Leiria, Portugal, dive into ways to automate GNS3 topology configurations, which, as will be seen in chapter~\ref{ch:gns3}, can be a major issue.
The usefulness of GNS3 for distance learning is also explored, for example, in~\cite{networkvirtwithgns}.

This section was just a brief presentation of GNS3, and is consequently incomplete with regard to technical and functional aspects, as well as on its mode of operation, as it will be examined in much more detail further down in this document.

% end of section exemulgns3


% Section "Kathará"
\input{Sources/Ch03_ExamplesOfEmulators/Sections/03_Kathara}

% Section "CORE"
\section{CORE}
\label{sec:exemulcore}

The CORE~\cite{coreemulator} emulator was developed originally by Jeff Ahrenholz at the Boeing Company.
Originally, CORE was an acronym meaning Common Open Research Emulator.
Its official website\footnote{\url{https://www.nrl.navy.mil/itd/ncs/products/core}} is hosted under the Networks and Communication Systems Branch of the U.S. Naval Research Laboratory, the institution currently responsible by its maintenance~\cite{Peach2016AnOO}.

It is a fork of a pre-existing project, Integrated Multiprotocol Network Emulator/Simulator (IMUNES), from the University of Zagreb.

It's described as \textquote{a tool for building virtual networks. As an emulator, CORE builds a representation of a real computer network that runs in real time, as opposed to simulation, where abstract models are used.}
Networks emulated with it can be, via the host machine of the emulation, connected to real world, physical networks.
Its documentation~\cite{coreghdocs} also states that \textquote{it provides an environment for running real applications and protocols, taking advantage of tools provided by the Linux operating system.}

Like the other emulators studied in this dissertation, except for GNS3, CORE is a framework for leveraging very lightweight OS-virtualized hosts, also known as containers, as a means to run isolated network stacks for each node---be it routers/switches, hosts, or any network service that can be run on top of Linux.

It is particularly similar to Kathará in that it offers a very wide range of services (essentally, appliances) working out of the box right after being added to a topology, relying on specific combinations of open-source software running on each node.

On the other hand, like GNS3, it provides a \gls{gui} as the default fashion to edit topologies and control a running emulation work session in real time, rather than a pure textual/declarative language to specify and configure projects (topologies) and a command-line interface to control the emulation~\cite{coreghdocs}.

% The first versions of CORE, like its predecessor IMUNES, were based upon both FreeBSD' networking stack and ``jails'' (a particular implementation of containers by that BSD operating system~\cite{freebsdjails})~\cite{comparisonofcore}.

% Nowadays, though CORE relies on one of the ways provided by the Linux kernel to enable ``containerization,'' which, as seen earlier in this text, is a way to isolate, at some degree, processes from each other, in this case with special attention to the IO and in particular the OS's networking stack. % TODO acronym for OS

% CORE comprises multiple parts, but from a high-level standpoint what matters the most is that it has offers a \gls{gui} as the main, official way to interact with the system, and then has a daemon running as a service in the host's Linux setup that is its backend which ensures the creation of namespaces and orchestrates the virtual bridges between them (which support the emulated links).

% It is documented to be able to run in a distributed fashion, like GNS3 (cf. ~\ref{sec:exemulgns3}), so that large topologies with a lot of nodes can be put set in motion, distributing that load of virtual nodes across multiple hosts.

% The support for a regular layer-3 network with standard routers is provided by Quagga~\cite{quagga}.
% All nodes are abstracted as GNU/Linux hosts, which can have a series of \emph{services} running.
% The set of services running by default depends on the type (a PC, layer-3 router, layer-2 switch, etc.) that is selected for each node in a project, but can be further customized.
% The full list of the services available can be consulted in~\cite{coreemuservices}, but we highlight. % TODO preencher isto e pôr uma tabela

% end of section exemulcore


% Section "Summary"
\section{Summary}
\label{sec:exemulsummary}

In this chapter we gave an overview of several network emulators, namely Mininet, GNS3, Kathará (while mentioning its predecessor, Netkit), and CORE.

In the next three chapters, we will dive deep in the analysis of functional and non-functional aspects, and also practical usage tests, for the last three of the aforementioned emulation tools.

% end of section exemulsummary


% end of chapter
