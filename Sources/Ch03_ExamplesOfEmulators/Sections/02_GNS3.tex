\section{GNS3}
\label{sec:exemulgns3}

GNS3, whose original name was ``Graphical Network Simulator-3,'' is a software project, comprising several distinct components which, despite the ``simulator'' in its name, falls under the category of emulators (cf.~section~\ref{sec:leavingemulationandsimulation}).

The GNS3 project was co-created by Jeremy Grossmann as a University project, at the École Informatique Epitech, part of the \emph{EPITECH Innovative Project (EIP)}.\footnote{\url{https://eip.epitech.eu/2013/gns3/en/index.html}, accessed on December 2019}
There is also a blog,\footnote{\url{http://gns3.blogspot.com/2007/}, accessed on December 2019} whose first posts date back to 2007, that announces the first ``beta'' releases of the software and gives details about the inception of GNS3, and lists its original developers.

Unlike Mininet, CORE, and Kathará, GNS3 was created with the goal of providing as much as possible the ability to create rich and heterogeneous topologies capable of running commercially available software (such as Cisco IOS), in GNS3 emulated switching and routing nodes, and was not particularly focused on being high performance and/or lightweight.
GNS3's list of capabilities also include the ability to run full-blown virtual machines, e.g. serving as hosts, and perform \gls{nat} and firewalling functions.

All of this is possible because GNS3's major strength is high functional realism and compliance with industrial production environments---making it a natural choice of students training for Cisco CCNA and related certifications~\cite{thebookofgns3,gns3netsimguide}---, as well as because of a clever architecture for load distribution that makes it relatively simple to scale high-resource demanding large and complex topologies and experiments.

The magnitude of the project, part of which comes from its inherent extensibility, has made it suitable for a myriad of use-cases.
For example, in the industry, GNS3 naturally can be a handy tool to prototype and test a topology for a real organization network, in the context of the practice of a network professional.
But also, in the age of the DevOps paradigm---a set of practices and methodologies for faster delivery of applications and services---, of which \emph{infrastructure as code} is one of the culprits~\cite{awswhatisdevops}, GNS3 can be of value for software engineers to test distributed applications that depend on certain network behaviors.

There is already some published work on the use GNS3 in education.
Usually, though, the experiments and comparisons are made against other ``graphical simulators''---instead of emulators---, since those are the software tools that, from a high-level perspective, are equivalent, even if the functional possibilities and implementations are completely different, if not opposite.

A brief comparison with Boson NetSim and Packet Tracer, two graphical simulators, is presented in~\cite{virtlabgnsvmware}---the authors explaining a way to use VMware Workstation on a desktop to deploy \glspl{vm} on GNS3's virtual topologies, acting as hosts, and run applications which communicate with each other over the emulated links.

In~\cite{automaticnetconfiggns}, the authors, two Professors from the Polytechnic Institute of Leiria, Portugal, dive into ways to automate GNS3 topology configurations, which, as will be seen in chapter~\ref{ch:gns3}, can be a major issue.
The usefulness of GNS3 for distance learning is also explored, for example, in~\cite{networkvirtwithgns}.

This section was just a brief presentation of GNS3, and is consequently incomplete with regard to technical and functional aspects, as well as on its mode of operation, as it will be examined in much more detail further down in this document.

% end of section exemulgns3
