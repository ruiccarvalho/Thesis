\section{CORE}
\label{sec:exemulcore}

The CORE~\cite{coreemulator} emulator was developed originally by Jeff Ahrenholz at the Boeing Company.
Originally, CORE was an acronym meaning Common Open Research Emulator.
Its official website\footnote{\url{https://www.nrl.navy.mil/itd/ncs/products/core}} is hosted under the Networks and Communication Systems Branch of the U.S. Naval Research Laboratory, the institution currently responsible by its maintenance~\cite{Peach2016AnOO}.

It is a fork of a pre-existing project, Integrated Multiprotocol Network Emulator/Simulator (IMUNES), from the University of Zagreb.

It's described as \textquote{a tool for building virtual networks. As an emulator, CORE builds a representation of a real computer network that runs in real time, as opposed to simulation, where abstract models are used.}
Networks emulated with it can be, via the host machine of the emulation, connected to real world, physical networks.
Its documentation~\cite{coreghdocs} also states that \textquote{it provides an environment for running real applications and protocols, taking advantage of tools provided by the Linux operating system.}

Like the other emulators studied in this dissertation, except for GNS3, CORE is a framework for leveraging very lightweight OS-virtualized hosts, also known as containers, as a means to run isolated network stacks for each node---be it routers/switches, hosts, or any network service that can be run on top of Linux.

It is particularly similar to Kathará in that it offers a very wide range of services (essentally, appliances) working out of the box right after being added to a topology, relying on specific combinations of open-source software running on each node.

On the other hand, like GNS3, it provides a \gls{gui} as the default fashion to edit topologies and control a running emulation work session in real time, rather than a pure textual/declarative language to specify and configure projects (topologies) and a command-line interface to control the emulation~\cite{coreghdocs}.

A particular trait of CORE which distinguishes it from the other simulators is its readily available support a special kind of wireless networks emulation/simulation, which may take into consideration, for routing purposes, the geometrical position of the nodes in the topology canvas and, according to a user-set scale, translates it into a geographical disposition, which can in turn produce different results~\cite{ospfmanet}.

% The first versions of CORE, like its predecessor IMUNES, were based upon both FreeBSD' networking stack and ``jails'' (a particular implementation of containers by that BSD operating system~\cite{freebsdjails})~\cite{comparisonofcore}.

% Nowadays, though CORE relies on one of the ways provided by the Linux kernel to enable ``containerization,'' which, as seen earlier in this text, is a way to isolate, at some degree, processes from each other, in this case with special attention to the IO and in particular the OS's networking stack. % TODO acronym for OS

% CORE comprises multiple parts, but from a high-level standpoint what matters the most is that it has offers a \gls{gui} as the main, official way to interact with the system, and then has a daemon running as a service in the host's Linux setup that is its backend which ensures the creation of namespaces and orchestrates the virtual bridges between them (which support the emulated links).

% It is documented to be able to run in a distributed fashion, like GNS3 (cf. ~\ref{sec:exemulgns3}), so that large topologies with a lot of nodes can be put set in motion, distributing that load of virtual nodes across multiple hosts.

% The support for a regular layer-3 network with standard routers is provided by Quagga~\cite{quagga}.
% All nodes are abstracted as GNU/Linux hosts, which can have a series of \emph{services} running.
% The set of services running by default depends on the type (a PC, layer-3 router, layer-2 switch, etc.) that is selected for each node in a project, but can be further customized.
% The full list of the services available can be consulted in~\cite{coreemuservices}, but we highlight. % TODO preencher isto e pôr uma tabela

% end of section exemulcore
