\section{GNS3}
\label{sec:exemulgns3}

% \subsubsection{Usage of GNS3 in education}
% \label{subsubsec:relgns3usageedu}

% There is already some published work about using GNS3 in education institutions.
% Usually, though, the experiments and comparisons are made against other ``graphical simulators''---instead of emulators---, since those are the software tools that, from a high-level perspective are equivalent, even if the functional possibilities and implementations are completely different, if not opposite.

% A brief comparison with Boson NetSim and Packet Tracer, two graphical simulators, is presented in~\cite{virtlabgnsvmware}, where the authors explain a way to use VMware Workstation VMs on a desktop to ``emulate'' end hosts on GNS3's virtual topologies, and using applications on those VMs which communicate with each other over the emulated links.

% In~\cite{automaticnetconfiggns}, the authors, two Professors from the Polytechnic Institute of Leiria, Portugal, dive into ways to automate GNS3 topology configurations, which, as will be seen in chapter~\ref{ch:gns3}, can be a major issue. % TODO

% The usefulness for distance learning is also explored, for example, in~\cite{networkvirtwithgns}.

% end of section exemulgns3
