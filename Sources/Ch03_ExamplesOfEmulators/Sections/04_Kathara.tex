\section{Kathará}
\label{sec:exemulkathara}

Kathará, presented as \textquote{Container-Based Framework for Implementing Network Function Virtualization and Software Defined Networks,} is an emulation framework developed at the Roma Tre University~\cite{kathara}.
It was created with the purpose of building a tool that, simultaneously, allowed to prototype and research with concepts considered ``recent'' or tending in networking and related fields.
% Its stated goal is to be able to prototype, test, learn, and research with SDN (via good OpenFlow switch support) and NFV (using ClickOS\footnote{ClickOS is a software system to provide Network Functions on lightweight middleboxes}~\cite{clickos}), and modern containers, namely Docker, as well as programmable data plane-enabled switches, using the new P4 domain-specific purpose programming language~\cite{p4programming}. % TODO add the concept of 'middlebox' to the glossary

According to its developers, Kathará was built having in mind the creation of software framework, a computer network emulator, that allowed to prototype, develop over, and experiment with, not only traditional switched and routed networks, but also with four emerging concepts and technologies in the scope of (or related to) networking:
\begin{itemize}
	\item SDN, enabling OpenFlow switches and their corresponding controllers to be added to topologies, allowing for separate data-plane-control-plane devices.
	\item NFV, offering good support for ClickOS~\cite{clickos} (a software system to provide Network Functions on lightweight middleboxes) nodes which can provide VNFs such as NATs, firewalls or DMZs.
	\item P4~\cite{p4programming}, a domain specific language that is compiled and run on switching/routing devices providing the concept of a ``programmable data-plane'' (in the node, instead of separately, as in the ``original'' SDNs)
	\item Containers (and, in particular, Docker) as the \textquote{lightweight and multi-platform alternative to standard virtualization solutions \ldots\ an isolated environment in which processes run independently, avoiding the overhead introduced by full virtualization that needs to maintain multiple kernels.}
\end{itemize}

However, despite this motivation, Kathará implements the same topology definition syntax and user interface that Netkit~\cite{netkit-short} has, an emulator developed years before at the same institution, does, and supports, from a high-level perspective, the same capabilities for experimenting with traditional networks comprising distributed routing devices.
Since this is one of the emulators being studied in more detail in the present work, no technical underpinnings are being presented in this section, only high-level features.

Unlike GNS3, Kathará doesn't follow a ``graphical first'' approach to its UI, since it doesn't offer, out-of-the-box, any visual topology editor nor allows to manipulate running labs via graphical controls.
Instead, it privileges a declarative approach for the definition of the labs/topologies as well as the usage of a command-line interface to start, stop, and interact with nodes.
In this sense, it is relatively similar to Mininet, but it leverages the decision of using Docker containers to provide a very ``extensible-by-design'' functional philosophy, instead of being geared towards a strict OpenFlow switch, controller, and end host.
Kathará explicitly wants to provide a rich environment to combine the emulation of latest software-defined and highly virtualized networks with the ``traditional'' model that Netkit already used to provide.

Kathará is essentially an academic tool.
As such, it is not oriented towards industry or vendor specific details, offering its exclusively through free and/or (partially) open-source software.

% end of section exemulkathara
