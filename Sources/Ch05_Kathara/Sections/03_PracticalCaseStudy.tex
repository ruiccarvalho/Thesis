\section{Using Kathará. Practical case study}
\label{sec:katharapracticalcasestudy}

In Kathará's founding paper, the described technical infrastructure of the program, which can easily be confirmed by consulting the available Python source code,\footnote{\url{https://github.com/KatharaFramework/Kathara}}, clearly states the basic design decision that follows the choosing of lightweight containers via Docker: every node in the topology is a Docker container.
Kathará, therefore, works as an orchestrator, the same way that Docker's own Docker Composer is.
The difference is that, while Docker Composer exposes a limited set of network parameterization, given that its purpose is to handle the combination of different application-level services running in containers that must be able to exchange IP traffic between each other, Kathará uses lower-level primitives exposed by Docker itself to setup the containers' virtual interfaces and hook them up to virtual collision domains which serve as a shared medium between any number of nodes, and therefore constitute links or hubs in a topology.

Kathará also leverages the notion of Docker images, available via Docker Hub or just setup locally on the user's Docker installation, both for separating the functionality (and ``role'') that different kinds of nodes may have on a lab (from being a traditional router to an OpenFlow switch, or an end-host running a RDBMS). % TODO add RDBMS to the acronyms

% end of section katharapracticalcasestudy
