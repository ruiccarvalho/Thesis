\section{Using Kathará. Practical Case Study}
\label{sec:katharapracticalcasestudy}

For the practical case study of Kathará, we followed the same approach than for GNS3's, used the same laboratory handouts from APRC as reference, but implemented in Kathará's different paradigm.
Refer to section~\ref{sec:gns3practicalcasestudy} for an overview of the labs and their topics.

\subsection{Setting Topologies in Kathará}

As mentioned earlier in this chapter, Kathará doesn't offer a graphical interface to draw topologies, and nodes added and removed, started and stopped, and text-based consoles to them are created via a set shell commands.

We mentioned as well that, in Kathará, there aren't end-to-end links as such, like there are on GNS3.
Instead, Kathará has collision domains, which are Ethernet media, shared by whichever interfaces are attached to them, and all hosts connected through a collision domain can have packets sent back and forth among each other.
For instance, each IP-prefix corresponding to a logical LAN, connected to a layer-3 router, can have all its hosts attached to the same collision domain.

In spite of the aforementioned, to be coherent with what was done in the GNS3 case-study and to be able to compare both frameworks more fairly---and because it is a condition to perform the switching assignment, in lab handout~2---it was decided that we would always use point-to-point links, which in practice means creating as many collision domains as there are links, and attaching each corresponding pair of interfaces in the nodes to every correct one. % TODO este point-to-point está certo?

% Figure fig:kathara-annotated-lab3-topology
\begin{figure}
  \centering
  \includegraphics[width=\textwidth]{kathara-annotated-lab3-topology}
  \caption{Section of ``lab-assignment3'' showing the topology, with every collision domain annotated}
  \label{fig:kathara-annotated-lab3-topology}
\end{figure}


To facilitate this process, the graphical representation of the topology as given in the course's handout PDF was annotated with the letters that are becoming each collision domain, as figure~\ref{fig:kathara-annotated-lab3-topology} depicts. % TODO fazer versão vectorial desta figura
Then, writing Kathará's \texttt{lab.conf} and making sure that it is correct becomes much easier.

\subsection{Switching Exercises}

The switching lab was set up using a similar methodology than the one described just before for lab~3---i.e. the topology was graphically annotated to define the collision domains among the several interfaces.

Then, using Kathará's convention to ``pass-on'' configuration directories to each container, we defined the startup file (that is executed by its respective container when it ``boots up'') to setup a bridge virtual interface, according to the Linux bridge documentation and online resources~\cite{brctlman,howtobridgelinux}.

The configurations, both the \texttt{lab.conf}, which defines the topology, and one of the nine (the other ones are analogous and only vary in the number of interfaces) \texttt{<node\_name>.startup}, are shown as examples in the listings appendix as listings~\ref{listlab2conf} and~\ref{listlab2startup}, respectively.

For the switching exercises, we found out that there is some kind of limitation, either in Docker itself, in Docker's chosen intercontainer networking strategy, or in Kathará's implementation.
The containers (lightweight virtual Linux hosts), equipped with multiple \glspl{nic} that perform each switch, are not capable of successfully performing the Spanning Tree Protocol, which is demanding for a topology with a loop, which the exercise deliberately has.
We were not able to find a convincing explanation for this fact.

\subsection{Working Under Load and Version Control}

Kathará doesn't provide any \textbf{load balancing} mechanism out of the box.
Containers are expected to run under one single Docker host.
On the one hand, one can try to imagine ways to tunnel the traffic from certain nodes to any arbitrary different topology, by bridging virtual Docker interfaces to some of the hosts' interface.
On the other hand, as exposed in more detail later on, Kathará doesn't demand nearly as many resources for a large topology as GNS3.

Regarding \textbf{version control} (what, in GNS3, is achieved via \textbf{snapshots}), Kathará's approach is different.
Every relevant configuration is expected to be written, either in the configurations passed in conventionalized sub-directories of the project to the hosts to configure the routing daemons or any other software running in the container, or in the declaration of the graph in the \texttt{lab.conf}.
All of these are plain-text files, therefore the user is encouraged to resort to a version-controls solution, like Git\footnote{\url{https://git-scm.com}}.

\subsection{Performance measurements}

For completeness, we measured the host to host bandwidth in Kathará, measured with \texttt{iperf3}, like shown, in the case of GNS3, in~\ref{tab:gns3-iperf}.
The values are consistently~$\approx 25.0~\mbox{Gbit/s}$, even on 1~Gbps virtual \glspl{nic}.

A measure of the OSPF convergence time for this emulator was not performed.

% end of section katharapracticalcasestudy
