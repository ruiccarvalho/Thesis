\section{Summary}
\label{sec:katharasummary}

This chapter presented Kathará as an emulator:
\begin{itemize}
  \item optimized for a use in academic environments, where vendor-specific aspects are not particularly relevant, only the implementation of protocols and networking functionality;
  \item very little resource-hungry due to its reliance on Docker containers, which are lightweight by nature;
  \item which offers a declarative, textual interface, both for topology definition and default node configuration, as well as interaction with an emulation in execution, through shell commands;
  \item able to very large topologies in one single host.
\end{itemize}

% end of section gns3summary
