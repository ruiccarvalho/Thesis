% !TEX root = ../Thesis.tex
% !TEX spellcheck = en-US

\chapter{Leaving the physical world}
\label{ch:leavingthephysicalworld}

% Section "Emulation and simulation"
\section{Emulation and simulation}
\label{sec:leavingemulationandsimulation}

Providing a precise definition, or ``definition framework,'' for distinguishing simulators from emulators and why \emph{emulation} (as opposed to \emph{simulation}) is the focus of this dissertation is paramount.
In the literature, Wikipedia, or common dictionaries, there are contradictory definitions, overlapping concepts, and, in general, a vague distinction between the two is presented~\cite{netsimoremu}. % TODO find and insert citations
That is why, rather than universally establishing what an emulator and a simulator are, it is intended that when, throughout this text, software package $A$ is an emulator and $B$ is a simulator in this or that sense, the reader can know without ambiguity what is meant.

% \begin{figure}
%   \centering
%   \includegraphics[width=0.8\textwidth]{emulationvsreality}
%   \caption{Simulation vs emulation vs reality}
%   \label{fig:emulationvsreality}
% \end{figure}

Before pointing out the differences between emulators and simulators, it is important to say what both have in common: they are ways of providing students, researches, developers, administrators, or any professional that works with computer networks with an alternative to the real network that is being studied, for convenience and/or cost effectiveness.
However, a lab with network nodes is also different from the reality it is trying to model.
%That idea is exposed in the diagram in figure~\ref{fig:emulationvsreality}.

On one end there is reality---in a lab, factors like amount of nodes, physical length of links (and therefore propagation times) can only be \textbf{simulated}, let alone on an solution that is fully software-based, like the emulators described later in this section.
An anecdote like the 500-mile email~\footnote{The case of the 500-mile email~\cite{500mileemail} was an issue described by a systems administrator where e-mails could not be sent to servers further than 500 miles.
The description seemed impossible as e-mail servers (closer) were working, but a hidden configuration---a timeout---rendered a different behavior according to the physical distance between hosts, due to round-trip times.
} could not be reproduced, without simulating an---then---unknown ``flaw,'' in an emulator consisting in virtual machines running the real software and performing the real operations: the cause of the problem was latency.
On the other end, there are pure simulators, defined afterwards.

% In between, complex solutions like GNS3, which is presented in great detail in chapter~\ref{ch:gns3}, can offer a mixture, while being mainly emulators in the sense that it consists in an orchestrated set of processes running the real code, and also allowing to simulate aspects of reality like failures in links and delays.

% The reason why emulators are specially interesting and are essentially the motivation (\ref{sec:motivation}) of this work, though, is that giving away the lab room with its switches and cables requires to not lose the flexibility it offers to perform all kind of pedagogical exercises, and ``pure'' simulators do not offer by inherent limitation.

%A network simulator is a computer program (or software package) that implements algorithms corresponding to a \textbf{model} of the reality through software components.

In~\cite{netkit-full}, the article presenting Netkit (a tool studied later in this thesis) and the process that led to its inception, there's an interesting paragraph that explains succinctly the difference between simulation and emulation.
Note that these concepts exist for any kind of \emph{system}, of which any computer system, such as computer networks, are only particular instances (emphasis \textbf{not} present in the original document):
\begin{displayquote}
Simulation aims at computing the behavior of a network based on an abstract model that usually ignores many functional details (e.g., message formats, the need for configurations, and the presence of command line interfaces).
This approach is suited for massive performance-oriented experiments and usually produces accurate event timing information.
Emulation aims at accurately reproducing the behavior of a real network in all its functional details and is usually achieved by exploiting as much as possible the same software (or an open version of it) that would be used on real devices.
Although considerably less efficient than simulation and usually inadequate to reproduce timings observed in real network systems, this approach is very effective to capture the wide range of functionalities and behaviors of a real network, and we believe \emph{it is therefore much more suited for didactics than simulation}.
\end{displayquote}
Simulators like ns-3~\cite{ns3} or the OPNET~\cite{introtoopnet} are discrete event simulators~\cite{netsimoremu}.
They enable, through some interface---such as a graphical one---, designing, testing, analyzing, and observing the expected behavior of protocols, nodes, and links according to several parameters. Therefore, it serves as a way to study, do research on, and even develop, protocols used across the stack.

Even though the usefulness of these tools cannot be questioned, and some like Cisco Packet Tracer,\footnote{\url{https://www.netacad.com/courses/packet-tracer/introduction-packet-tracer}} also a discrete event simulator~\cite{evaluatingnetsimmethodologicapproach}, are very widely used in the training of professionals in the networks administration and engineering fields~\cite{rolepackettracer}, it is crucial to point out as the distinctive characteristic of simulators, as defined here, the fact that those applications constitute a ``closed'' universe in the sense of only \emph{simulating} events that change the status of internal data structures, but do not implement a real stack of protocols, where every operation that should be performed is, in fact, performed. % TODO this is super "Portuguese in English". Improve

In other words: a network simulator is a black box for the outside, with a set of available inspection and state alteration operations---i.e. that provides the ability to see some aspects of the simulated system and perform some finite set of changes to the simulated environment.
% However, as long as the \emph{implemented} cause-effect behaviors for that set of operations \emph{is coherent} to their real-world counterparts, any kind of ``shortcuts'' inside can be taken.

On the other hand, the definition for an emulator can, therefore, pretty much be inferred from the simulator as its ``complementary.''
Software computer network emulators are solutions that allow, without resorting to the usage of physical equipment (even though sometimes allow for it, if desired) prototyping networks for analysis and study of underlying mechanisms, investigation, and development of new protocols and solutions, etc., thanks to an array of techniques that can span from virtualization and containerization, hardware architecture emulation, and leveraging operating system functionality in multi-process environments.

Unlike the simulators, described earlier, the fact of the traffic being real (at least from a logical standpoint) and that is possible to ``glue'' emulated interfaces with any kind of physical ones detected by the host operating system, offers the possibility of mixing emulated topologies with real topologies, cloud machines, etc.
Besides, since the computational nodes (hosts in the emulated topologies) are in, fact, ``generic'' hosts (nowadays, usually VMs or containers), running real software, the limit in terms of what it is possible to configure, parameterize, implement, or test, for any stack level (from the link to the application layer) is virtually nonexistent.

However, with any software based solution there is also a virtualization of the physical layer.
Sometimes, that virtualization is not taken into account, since there is no obligation of layer 2 to be supported on any kind of physical link (it can be mere in memory communication), but, since many aspects of networking address physical characteristics of some links, when necessary, these attributes (like distance induced latency or packet loss) have to be simulated in code by whatever mechanism is supporting the links.

% end of section leavingemulationandsimulation


% \section{Notable simulators}
% \label{sec:notablsimulators}

% Packet Tracer, Boson NetSim, ns-3...

% \section{Notable emulators}
% \label{sec:notablemulators}

% \subsection{GNS3}
% \label{subsec:relworkgns3}

% \subsubsection{Usage of GNS3 in education}
% \label{subsubsec:relgns3usageedu}

% There is already some published work about using GNS3 in education institutions.
% Usually, though, the experiments and comparisons are made against other ``graphical simulators''---instead of emulators---, since those are the software tools that, from a high-level perspective are equivalent, even if the functional possibilities and implementations are completely different, if not opposite.

% A brief comparison with Boson NetSim and Packet Tracer, two graphical simulators, is presented in~\cite{virtlabgnsvmware}, where the authors explain a way to use VMware Workstation VMs on a desktop to ``emulate'' end hosts on GNS3's virtual topologies, and using applications on those VMs which communicate with each other over the emulated links.

% In~\cite{automaticnetconfiggns}, the authors, two Professors from the Polytechnic Institute of Leiria, Portugal, dive into ways to automate GNS3 topology configurations, which, as will be seen in chapter~\ref{ch:gns3}, can be a major issue. % TODO

% The usefulness for distance learning is also explored, for example, in~\cite{networkvirtwithgns}.

% \subsection{Kathara}
% \label{subsec:relworkkathara}

% \subsection{CORE}
% \label{subsec:relworkcore}

% \subsection{Mininet}
% \label{subsec:relworkmininet}

% \section{Usage of simulators and emulators in teaching}
% \label{sec:simemulusage}

% end of chapter
